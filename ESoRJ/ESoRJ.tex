\documentclass[11pt, a4paper, notitlepage]{article}
\usepackage{graphicx}   
\usepackage{wrapfig}  
\usepackage{url}
\usepackage[margin=3cm]{geometry}

\title{Mathematics into School - Executive Summary of Reflective Journal}
\author{Tom Mann}
\date{\today}

\begin{document}

\maketitle

\section*{Introduction}
I spent 9 weeks at Our Lady and st Thomas catholic primary school, and as part of my time there I delivered 6 sessions to a group of girls in the year 6 class. During my time in the classroom and running my sessions I had many of my preconceptions around teaching challenged. I would like to discuss these, starting with the use of competition in the class, including both students competing against themselves and against other students.

\section*{Competition in the Classroom}
During the later weeks of my placement I observed the class I was in playing a "times tables league". This consisted of the students completing a 12 by 12 grid of times tables against the clock (Journal, Week 16). They then read out their times and they were rewarded points based on their time, with a bonus point for moving down into a new minute bracket. Their scores were kept track of over the year, with a prize awarded to people with the most points at the end of each term. I initially thought that this kind of competition was  detrimental to students learning due to a few reasons. Firstly, some students may become demotivated as it is impossible for them to catch up to the students with the most points. Secondly, competition may increase students fear of failure or lead to low self-esteem. 
\par
One study found that students in "competitively structured discussion" are more anxious and lose self-assurance \cite{Roger:1973}. This is something my observations from the classroom agree with. Earlier in my placement the students took a 10 minute arithmetic test, where similarly their scores where read out. Afterwards I heard two students exchange "What did you get?", followed by "I don't know" (Journal, Week 12). This clearly shows how even when students aren't directly competing, simply having their score shared with their peers can make them feel anxious. 
\par
In my last week at the school, one of the more maths-confident students asked to do a "Times tables league", to my surprise when it was put to a vote the vast majority of the class wanted to take part (Journal, Week 19). It appears that competition does act as an effective form of motivation for the children. Shindler agrees that competition is an extremely powerful tool for motivation, but there are features of competition that make it healthy or unhealthy \cite{Shindler:2009}. Most importantly a competition must be short term, have no real or significant reward, all individuals have a reasonable chance of winning and the goal is primarily fun. The "Times tables league" met some of these criteria, but the combination of the long term points system and it being almost impossible for some students to win pushes it closer to unhealthy competition. 
\par
One study of undergraduates found that introducing a competitive element to game based learning improves learning outcomes and motivation of participants \cite{Cagiltay:2015}. I hoped to recreate the benefits seen in this study in my last session with the students. The session involved the students playing a simple game against me and their peers, importantly their was no prize, the game was simple enough that all students could win and the competition was only 20 minutes. This proved extremely effective as some of the more uninterested students put more effort into this session than previous ones (Journal, Week 19). The two main things that differentiated this from standard competition is that the students gained something when they lost, and the only thing the students were competing for was the pride of winning. As the main aim of the session was to develop a strategy for the game, the children learn by losing which likely took the fear of failure away. A standard competition involves a resource or goal that cannot be shared by everyone competing, in this case this is only the pride of winning, which is minimal as the competition was so short.
\par
% TODO: Add another reference


\section*{How to Write a Good Question}
%Questions are usually only thought about as a way to examine knowledge not a way to teach, but a good question can help students understanding and a bad question can set students back. It was in my second week at the school that I noticed the first badly written questions given to the students. The students were studying ratios, and were given a question similar to "If it takes 6 builders 9 days to build a kitchen, how many days would it take 2 builders?" (Journal, Week 12). While there is nothing inherently wrong with this question, none of the children answered it correctly showing that it did not provide 

\section*{Girls \& Boys}
\section*{Conclusion}

\bibliographystyle{unsrt}
\bibliography{bibliography}

\end{document}