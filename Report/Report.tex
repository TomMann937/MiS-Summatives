\documentclass[11pt, a4paper, notitlepage]{article}
\usepackage{graphicx}   
\usepackage{wrapfig}  
\usepackage{url}
\usepackage[margin=3cm]{geometry}

\title{Mathematics into School - Project Report}
\author{Tom Mann}
\date{\today}

\begin{document}

\maketitle

\begin{abstract}
    This report details a project delivered in partnership with Our Lady and st Thomas Catholic primary school with the aim of increasing girls confidence in maths and decreasing cases of maths anxiety. Over a series of six session a group of girls were introduced to a variety of mathematical content. The pedagogical insight behind the mathematical content, design and delivery of these sessions will be examined. Finally the report will evaluate the success of the project in regards to confidence anxiety, along with the legacy to the students.
\end{abstract}

\clearpage

\tableofcontents

\clearpage

\section{Introduction}
This is what will go in the introduction

\subsection{Context}
This project was conducted in partnership with Our lady and st Thomas school (OLST), a  co-educational Catholic primary school consisting of around 120 students. The school has strong academic results with the percentage of students achieving expected or higher standard in reading, writing and maths is above the national and local average \cite{OLST_stats}. However it has been observed that some girl in the year six class lack confidence in maths lessons and interact less than hoped.


\subsection{Aims}
There were two main aims for the project, although they may appear similar and overlap in some aspects they are distinct.

\subsubsection{Aim 1 -- Increasing self-confidence in girls in maths} 
Confidence within any subject is the belief that you can rely on you own abilities. It appears that there are two ways to increase self-confidence; either improve students abilities or improve the belief in their own abilities. Nationally and in the North East the percentage of girls and boys achieving the expected standard in their maths SATs has been within 2 percentage points of each other since 2018/19 \cite{maths_SATs_stats}. This suggests that while increasing girls abilities in maths may increase their confidence this may not be the main cause of a lack of self confidence.
\subsubsection{Aim 2 -- Deacreasing maths anxiety in girls}

\section{Aims and Objectives}
This is what will go in the aims and objectives. 

\section{Design and Delivery}

This is what will go in design and delivery

\section{Evaluation}
This is what will go in the evaluation

\bibliographystyle{unsrt}
\bibliography{bibliography}

\end{document}