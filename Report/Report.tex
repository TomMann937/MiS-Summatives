\documentclass[11pt, a4paper, notitlepage]{article}
\usepackage{graphicx}   
\usepackage{wrapfig}  
\usepackage{url}
\usepackage[margin=3cm]{geometry}

\title{Mathematics into School - Project Report}
\author{Tom Mann}
\date{\today}

\begin{document}

\maketitle

\begin{abstract}
    This report details a project delivered in partnership with Our Lady and st Thomas Catholic primary school with the aim of increasing girls confidence in maths and decreasing cases of maths anxiety. Over a series of six session a group of girls were introduced to a variety of mathematical content. The pedagogical insight behind the mathematical content, design and delivery of these sessions will be examined. Finally the report will evaluate the success of the project in regards to confidence anxiety, along with the legacy to the students.
\end{abstract}

\clearpage

\tableofcontents

\clearpage

\section{Introduction}

\subsection{Context}
This project was conducted in partnership with Our lady and st Thomas school (OLST), a  co-educational Catholic primary school consisting of around 120 students. The school has strong academic results with the percentage of students achieving expected or higher standard in reading, writing and maths is above the national and local average \cite{OLST_stats}. However it has been observed that some girl in the year six class lack confidence in maths lessons and interact less than hoped.


\subsection{Aims}
There were two main aims for the project, although they may appear similar and overlap in some aspects they are distinct.

\subsubsection{Aim 1 -- Increasing self-confidence in girls in maths} 
Confidence within any subject is the belief that you can rely on you own abilities. It appears that there are two ways to increase self-confidence; either improve students abilities or improve the belief in their own abilities. Nationally and in the North East the percentage of girls and boys achieving the expected standard in their maths SATs has been within 2 percentage points of each other since 2018/19 \cite{maths_SATs_stats}. This suggests that while increasing girls abilities in maths may increase their confidence this may not be the main cause of a lack of self-confidence. One study \cite{Georgiou01122007} found two differences in the mindset of girls and boys. Firstly "girls tend to attribute failure in mathematics to a lack of ability whereas boys attribute failure to a lack of effort", and second girls are more likely to attribute success to external factors, such as an easy exam. This leads to a dichotomy where many girls take responsibility for their failures but do not take credit for their successes. So it was decided the best way to tackle self-confidence was in the mindset of girls.
\subsubsection{Aim 2 -- Deacreasing maths anxiety in girls}
Maths anxiety is in the broadest sense a feeling of worry or nervousness combined with physiological reactivity to current or future situations involving maths \cite{Luttenberger:2018}. While this is related to low self-confidence in maths, it is possible to have one without the other. Similarly to self-confidence girls seem to be disproportionately affected. The National Numeracy survey found that women were twice as likely to experience maths anxiety as men \cite{NationalNumeracy_anxiety}, however it is important to note general anxiety is also higher amongst girls. It has been found that there is a negative relationship between maths anxiety and maths performance, that is higher maths anxiety can lead to poor performance. There are various studies around the cause and effect relationship between maths anxiety and maths performance, it is unknown if one causes the other or many believe that the relationship is cyclic. This again means that one cannot simply treat maths anxiety by improving maths performance. When dealing with maths anxiety it is important to treat both the cause and the symptoms, treating the symptoms of anxiety would be more aligned with behavioural therapy, this report will focus on treating the cause of maths anxiety.

\section{Design and Delivery}
This is what will go in design and delivery

\section{Evaluation}
This is what will go in the evaluation

\bibliographystyle{unsrt}
\bibliography{bibliography}

\end{document}