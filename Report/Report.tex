\documentclass[11pt, a4paper, notitlepage]{article}
\usepackage{graphicx}   
\usepackage{wrapfig}  
\usepackage{url}
\usepackage{gensymb}
\usepackage[margin=3cm]{geometry}

\title{Mathematics into School - Project Report}
\author{Tom Mann}
\date{\today}

\begin{document}

\maketitle

\begin{abstract}
    This report details a project delivered in partnership with Our Lady and st Thomas Catholic primary school with the aim of increasing girls confidence in maths and decreasing cases of maths anxiety. Over a series of six session a group of girls were introduced to a variety of mathematical content. The pedagogical insight behind the mathematical content, design and delivery of these sessions will be examined. Finally the report will evaluate the success of the project in regards to confidence anxiety, along with the legacy to the students.
\end{abstract}

\tableofcontents

\clearpage

\section{Introduction}
% TODO: Add more to context
\subsection{Context}
This project was conducted in partnership with Our lady and st Thomas school (OLST), a  co-educational Catholic primary school consisting of around 120 students. The school has strong academic results with the percentage of students achieving expected or higher standard in reading, writing and maths is above the national and local average \cite{OLST_stats}. However it has been observed that some girl in the year six class lack confidence in maths lessons and interact less than hoped. 


\subsection{Aims}
There were two main aims for the project, although they may appear similar and overlap in some aspects they are distinct.

\subsubsection*{Aim 1 -- Increasing self-confidence in girls in maths} 
Confidence within any subject is the belief that you can rely on you own abilities. It appears that there are two ways to increase self-confidence; either improve students abilities or improve the belief in their own abilities. Nationally and in the North East the percentage of girls and boys achieving the expected standard in their maths SATs has been within 2 percentage points of each other since 2018/19 \cite{maths_SATs_stats}. This suggests that while increasing girls abilities in maths may increase their confidence this may not be the main cause of a lack of self-confidence. One study \cite{Georgiou01122007} found two differences in the mindset of girls and boys. Firstly "girls tend to attribute failure in mathematics to a lack of ability whereas boys attribute failure to a lack of effort", and second girls are more likely to attribute success to external factors, such as an easy exam. This leads to a dichotomy where many girls take responsibility for their failures but do not take credit for their successes. So it was decided the best way to tackle self-confidence was in the mindset of girls.
\subsubsection*{Aim 2 -- Decreasing maths anxiety in girls}
Maths anxiety is in the broadest sense a feeling of worry or nervousness combined with physiological reactivity to current or future situations involving maths \cite{Luttenberger:2018}. While this is related to low self-confidence in maths, it is possible to have one without the other. Similarly to self-confidence girls seem to be disproportionately affected. The National Numeracy survey found that women were twice as likely to experience maths anxiety as men \cite{NationalNumeracy_anxiety}, however it is important to note general anxiety is also higher amongst girls. It has been found that there is a negative relationship between maths anxiety and maths performance, that is higher maths anxiety can lead to poor performance. There are various studies around the cause and effect relationship between maths anxiety and maths performance, it is unknown if one causes the other or many believe that the relationship is cyclic. This again means that one cannot simply treat maths anxiety by improving maths performance. When dealing with maths anxiety it is important to treat both the cause and the symptoms, treating the symptoms of anxiety is more aligned with behavioural therapy, this report will focus on treating the cause of maths anxiety.

\subsection{Introduction to Students}
Due to the nature of the aims, this project is very individual. It was decided to to spend some time to learn the experiences of the students who would be taking part in the sessions. One session was spent talking to the girls in the class to find out more about what the prevalence of maths anxiety was amongst the girls and what are the causes of the maths anxiety. All of the students in the session agreed with each other that they had been worried in maths lessons before. It was clear that all of the students had experienced maths anxiety before it was not appropriate or necessary to find out more about individual experiences of anxiety, it was more important to find out the causes of the the anxiety.
\par
The girls identified three main causes of maths anxiety: fear of failure, judgement, and being left behind. One student added that maths feels frustrating because when they don't understand something, they feel stuck and unable to progress. It was with these insights in mind that the sessions were designed.

\section{Activities}
\subsection{Structure}
The year 6 class this project was taking part with had 8 girls in it, it was decided to include all of the girls from the class in the sessions. This may have reduced the amount of time that could be spent with each student, however in such a small class excluding only a few students would be unfair. 
\par It has been found that maths anxiety is often linked to teaching style, in particular in a traditional classroom setting the teacher takes an authoritative role \cite{Finlayson:2014}. To mediate this the sessions would be delivered around a group table with everyone sitting to reduce my presence as a directive role. 


\subsection{Session 2: M\"obius strips}
\subsubsection{Design and Delivery}
This session was designed with two strategies for dealing with maths anxiety in mind. The first stems again from the link between delivery methods and maths anxiety, in particular traditional delivery methods focus on the outcome of a lesson, whether that be students gaining specific knowledge or being able to answer certain questions correctly. The remedy to this is a more constructivist approach to teaching focusing on the process rather than the product \cite{Finlayson:2014}. The second strategy is focused on Expectancy-Value Theory, "ACER's Mathematics anxiety and Engagement Strategy" \cite{MAES:2024} outlines four different types of value that are important motivation, we shall focus on intrinsic value in this session and utility value later on. Intrinsic value comes simply from finding value in maths because you find it interesting or enjoyable.

\begin{wrapfigure}{r}{0.3\textwidth}
\includegraphics[width=0.25\textwidth]{Images/Möbius_Strip.jpg}
\caption{M\"obius strip}
\end{wrapfigure}
\par

To align with the goal of focusing on the process of learning, the first session was created with no learning outcome in mind, that is there is no expectations for the students to learn or recreate anything from the session. The intrinsic value of the session comes from exploring a shape the students will most certainly never have seen before, with properties they will never have seen before. 

\vspace{1em}

A M\"obius strip is created by taking a strip of paper, giving it a half-twist, and then joining the ends together. This creates a one sided, one edged surface. The session consisted of the students working in pairs to: make, draw on and cut the M\"obius strips. These aspects of the session align with two constructivist methods, group work and manipulation of materials as a primary source. 
\par
There was demonstration in front of the children of how to create a M\"obius strip, they then worked in pairs, sometimes with external help to construct their own M\"obius strip. The children were then told to draw length ways along the M\"obius strip until they get back to the start. This appears to end up with a line on "both" sides of the shape even though the students only drew one line. Through this task alone the students didn't realise the surface only had one side, so they were asked a slightly incorrect question of "You only drew on one side but there is a line on both sides, how is this possible?". Through constructing a destructing more M\"obius loops the students explored how a surface can have only one face.
\par
The next task involved the students cutting down the line they had drawn. Unexpectedly the surface does not end up in two pieces, but stays as one loop with more twists in it. This property was impossible for the students to miss when they completed this task. This property was more difficult to explain simply by building M\"obius strips, so in the session a prop was created by writing left and right at the end of the end of the strip so that left would align with left and right with right when the end were joined without a twist. The students could then observe that when we add a half-twist to make a M\"obius strip that left joins with right and vice versa. This showed that when they cut the strips there is no left and right side for them to split into. Note, here left and right do not have any meaning and are only to show that the sides do not join as expected. 

\subsection{Session 3/4: Statistics}

Whereas the previous activity was designed with intrinsic value in mind this activity, taking place over two lunch time sessions, was designed amongst other things, around utility value intervention. This is finding value in maths because you consider it useful for future or present goals \cite{MAES:2024}. The second goal of this session was to get students talking about maths with their parents. It has been shown that parental involvement can affect children's mathematics performance by reducing mathematics anxiety \cite{Vukovic01052013}. However other studies have found that children of maths anxious parents learn less maths over the school year and have more maths anxiety by the end of the year only if the parents report helping frequently with maths homework\cite{Maloney:2015}. As a result of this, involvement of parents was aimed to be based around discussion rather than maths questions.
\par
The topic of statistics was chosen as is on the national curriculum, so the students should have seen it before, and it offers many areas of application for the students to explore. This activity was split over two sessions so the students could collect data in between the sessions to use in the second session. The majority of the first session was spent discussing what statistics is and why it is useful. The students were then presented with a sheet for them to use to collect data, specifically hair colour, eye colour and age. 
% TODO: Add photo of data collection sheet
The aim was for students to take these sheets home and to collect data by speaking to their family. The sheet also contained a brief task for the students to research or speak to their parents about a use for statistics, incentivised by a reward of points from a system used in the school.
\par
\vspace{1em}
The second session focused on the students analysing the data they had collected, this was to be done through the students drawing bar charts and working out averages. A sheet of fake data was created for students who forgot their sheets. The session started with the students using examples of bar charts to see that they are easier to read off of than a table of data. The students then constructed their own bar charts for hair and eye colour on a sheet designed to offer guidance. This task was completed quicker than anticipated, so the session was adapted to teach the children about the median. The children lined themselves up in age order, and then the middle student was taken as example of a median of their ages. This again links to a constructivist approach to teaching deviating from a traditional delivery method, with the students themselves being a primary material to learn from.

\subsubsection*{Evaluation}
This task failed to meet the desired aims two main ways: the mode of delivery and promoting discussion around maths. Due to this session being focused on utility value, this meant trying to fit a larger amount of content into the sessions, and falling back to a more traditional delivery style. Specifically there was one authoritative role in the classroom and although sessions were discussion based the students were still completing worksheets. One student exclaimed "Ughh, not a worksheet" at the beginning of the first session. This shows how much a traditional delivery style is linked to maths anxiety or at least a lack of enjoyment.
\par
A large aim of this session was promoting discussion around maths with parents and family members through the data collection and researching statistics. As the data the students had to collect was so trivial (hair colour, eye colour and age), the students simply recalled the information and wrote it down on the sheet without speaking to anyone. Although some students showed interest in finding uses of statistics at the end of the first session, none of them completed any research on their own. The vast majority of students either lost the data they collected or forgot to bring it in, this means any motivation provided by the students working on their own data was lost.
\par
The rest of the sessions were designed to be self contained and to avoid overlapping with traditional delivery methods where possible.

\subsection{Session 5: Coordinate Grid}
\subsection{Session 6: 1-2 Nim}


\section{Evaluation}
This is what will go in the evaluation

\bibliographystyle{unsrt}
\bibliography{bibliography}

\end{document}